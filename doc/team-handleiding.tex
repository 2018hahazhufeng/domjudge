\documentclass[11pt,titlepage,a4paper]{article}

% $Id$

\usepackage{a4}
\usepackage[dutch]{babel}
\usepackage{amsmath,amssymb,amsthm}
\usepackage{epsfig}
\usepackage{bbm}
\usepackage{expdlist}

\setlength{\parindent}{0pt}
\setlength{\parskip}{1ex plus 0.5ex minus 0.2ex}

\newcommand{\DOMjudge}{DOMjudge }

\title{Teamhandleiding voor het \DOMjudge jurysysteem}

\author{Jaap Eldering, Thijs Kinkhorst \& Peter van de Werken}

\begin{document}

\maketitle

\section{Inleiding}

Deze handleiding beschrijft de onderdelen van het \DOMjudge
jurysysteem, die aan de teams ter beschikking staan om oplossingen in
te sturen, uitslagen te bekijken en ``clarification requests'' te
sturen.

Het systeem is voor een groot deel gebaseerd op een webinterface. Het
insturen van oplossingen gaat echter via een command-line programma.

\section{Oplossingen insturen: \texttt{submit}}

Het insturen van oplossingen voor problemen gebeurt via een
command-line interface: het programma \texttt{submit} (onder Windows
\texttt{submit.exe}).

\textbf{Syntax:} \texttt{submit [opties] bestandsnaam.ext}

Het submit programma haalt de naam van het probleem uit
\texttt{bestandsnaam} en de de programmeertaal uit de extensie
\texttt{ext}. Dit kan handmatig aangepast worden met de opties
\texttt{-p probleemnaam} en \texttt{-l taalextensie}. Zie
\texttt{submit --help} voor een compleet overzicht van mogelijke
opties en extensies en een aantal voorbeelden. Let op, dat deze help-tekst
redelijk lang is en misschien niet op \'e\'en scherm past. Gebruik dan
bijvoorbeeld \texttt{submit --help | more} om alles te lezen.

\texttt{submit} zal het bestand controleren op een aantal eigenschappen
en eventueel waarschuwingen geven, zoals wanneer het bestand al lange
tijd niet veranderd is of groter is dan de maximale source-code grootte.

Hierna geeft \texttt{submit} een kort overzicht met de details van de
inzending en vraagt om bevestiging. Controleer vooral of je het goede
bestand, probleem en taal hebt en druk dan op `y' om de oplossing in
te sturen. Als alles goed gaat zal \texttt{submit} een melding geven,
dat de inzending succesvol is. Indien niet, zal er een foutmelding
verschijnen.

Het submit programma maakt gebruik van een directory \texttt{.submit}
in de homedirectory van het account. Hier slaat het tijdelijk
bestanden op voor inzending en staat ook een logfile \texttt{submit.log}.
Verwijder deze directory niet of pas hem niet aan, omdat anders het
submit programma eventueel niet meer correct functioneert. Verder
is een ``public-ssh-key'' van de jury in de ssh config
toegevoegd. Deze is ook nodig voor het functioneren van \texttt{submit}.
 
\section{De uitslag bekijken van inzendingen}

\subsection{Mogelijke uitslagen}

Op een ingestuurde oplossing zijn een aantal verschillende uitslagen
mogelijk. Hier volgen ze, met een korte beschrijving:

\begin{description}[\setleftmargin{4.5cm}]
\item[CORRECT]
Je oplossing was goed en je hebt dit probleem opgelost!

\item[COMPILER-ERROR]
Het compilen van je programma gaf een fout. Zie de inzendingsdetails
voor de precieze foutmelding.

\item[TIMELIMIT]
Je programma draaide langer dan de maximale toegestane tijd en is
afgebroken. Dit kan betekenen dat je programma ergens in een loop
blijft hangen, of dat je oplossing niet effici\"ent genoeg is.

\item[RUN-ERROR]
Je programma gaf een fout tijdens het uitvoeren. Dit kan verschillende
oorzaken hebben, zoals deling door nul, incorrecte geheugen adressering
(segfault, bijvoorbeeld door arrays buiten bereik te indiceren), te
veel geheugen gebruik, enzovoort\dots
Let ook op, dat je programma met een exitcode nul eindigt!

\item[NO-OUTPUT]
Je programma gaf geen uitvoer. Let op dat je uitvoer naar standard
output schrijft!

\item[WRONG-ANSWER]
De uitvoer van je programma was niet correct.

\end{description}


\section{Clarification requests insturen}


\end{document}
