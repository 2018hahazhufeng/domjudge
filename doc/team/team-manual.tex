\documentclass[11pt,a4paper,twoside]{article}

\usepackage{a4wide}
\usepackage{moreverb}
\usepackage{graphicx}
\usepackage{wrapfig}
\usepackage{expdlist}
\usepackage{svn}
\usepackage{fancyhdr}
\usepackage{extramarks}
\usepackage{color}
\usepackage[colorlinks=true,urlcolor=blue,linkcolor=blue]{hyperref}
\usepackage{ifthen}

\pagestyle{fancy}
\renewcommand{\sectionmark}[1]{\markboth{\thesection.\ #1}{}}

% Begin new paragraphs without indentation but vertical space.
\setlength{\parindent}{0pt}
\setlength{\parskip}{1.5ex plus 0.5ex minus 0.2ex}

\newcommand{\DOMjudge}{\textsc{DOM}judge }

% Display commands, arguments, etc. in texttt font and don't break those.
\newcommand{\cmd}[1]{\mbox{\texttt{#1}}}

% Only show text if the submitclient is configured
\newcommand{\ifcmdsubmit}[1]{\ifthenelse{\equal{\SUBMITCLIENTENABLED}{yes}}{#1}{}}

% Our titlepage, should be called at start of team manual document
% Argument list:
% #1 - DOMjudge version
% #2 - Document revision
% #3 - Last modified date
% #4 - Generated date
% Also define the following words for language overrides:
\newcommand{\versionrevison}{Version/revision}
\newcommand{\lastmodified}{Last modified}
\newcommand{\generated}{Generated}
\makeatletter
\newcommand{\titlestuff}[4]{%

  \thispagestyle{plain}
  \vspace*{-3cm}
  \parbox[t]{\linewidth}{%
    \begin{wrapfigure}[1]{r}{2cm}
      \vspace*{-1cm}\hfill
      \includegraphics[height=4cm]{../logos/DOMjudgelogo.pdf}
    \end{wrapfigure}
    {\fontfamily{phv}\fontseries{b}\fontsize{26pt}{28pt}\selectfont \@title \par}
  }
  \vskip 2cm

  % Setup fancy headers/footers (here because we need SVN stuff defined)
  \def\setupfancystuff{%
    \fancyhead{}
    \fancyfoot{}
    \fancyfoot[RO,LE]{\thepage}
    \fancyfoot[LO,RE]{%
      \color[gray]{0.5}\vspace{-0.3cm}
      \begin{tabular}{ll}
        \versionrevison: & #1 / #2 \\
        \lastmodified:   & #3 \\
        \generated:      & #4 \\
      \end{tabular}
    }
  }

  % First for fancy page style:
  \setupfancystuff
  \fancyhead[RO,LE]{\slshape \firstleftmark}

  % No headers for plain page style (titlepage):
  \fancypagestyle{plain}{%
    \setupfancystuff
    \renewcommand{\headrulewidth}{0pt}
  }
}
\makeatother


\usepackage[english]{babel}

\title{Team manual \DOMjudge\\(version \DOMJUDGE_VERSION)}

\begin{document}

\domjudgetitlepage

\newpage
\tableofcontents

\newpage
\section{Introduction}

This document describes these parts of the \DOMjudge jury-system, that
are available to teams to submit solutions, view results and submit
``clarification requests''.

All interaction with the system is done using a web-interface, except
(optionally) the submission of solutions: this is done either in the
web-interface or with a command-line program. Availability of either
may depend on the configuration by the jury.

\section{Summary}

Here follows a short summary of the system interface. This is meant as
a quick introduction, to be able to start using the system. It is
however strongly advised, that at least one of your team's members
read all of this manual, because there are specific details of this
jury-system, that might become of importance when you run into
problems (like an incorrect solution). \textbf{BE WARNED!!}

\subsection{Reading and writing}

Solutions have to read all input from `standard in' and write all
output to `standard out' (also known as console). You will never have
to open (other) files. See appendix \ref{codeexamples} for some
examples.

\subsection{Submission of solutions}

You can submit solutions with the command-line program \cmd{submit} or
the using web-interface:
\begin{description}[\breaklabel\setlabelstyle{\bfseries}]
\item[Command-line] 
Use \cmd{submit <problem>.<extension>}, where \cmd{<problem>} is the
label of the problem and \cmd{<extension>} is a standard extension for
your language.  For a complete reference of all options and examples,
see \cmd{submit --help}.
\item[Web-interface]
From your teampage \url{\WEBBASEURI team}, browse to
\textbf{submit} and select the file you want to submit. By default,
the problem is selected from the base of the filename and the language
from the extension, just as with the command-line submit.
\end{description}

\subsection{Viewing scores, submissions, etc.}

Viewing scores, submissions and submitting and reading clarification
requests is done through a web-interface. The address is
\url{\WEBBASEURI team}. Usage of these webpages should be clear.

\newpage
\section{Submitting solutions}\label{submit}

Submitting solutions can be done in two ways: with the command-line
program \cmd{submit} (\cmd{submit.exe} in Windows) or using the
web-interface. One of the interfaces might not be available, depending on
the system configuration by the jury. A description of both methods
follows.

\subsection{Command-line: \cmd{submit}}

\textbf{Syntax:} \cmd{submit [options] filename.ext}

The submit program takes the name (label) of the problem from
\cmd{filename} and the programming language from the extension
\cmd{ext}. This can be overruled by hand with the options
\cmd{-p problemname} and \cmd{-l~languageextension}.
See \cmd{submit --help} for a complete list of all options and
extensions and some examples. Note that this help is quite long and
might not fit on a screen. Then use for example
\cmd{submit~--help | more} to read everything.

\cmd{submit} will check your file and warns you for some things:
for example when the file has not been modified for a long time or
when is is larger than the maximal source-code size.

Then \cmd{submit} displays a summary with all details of your
submission and asks for confirmation. Check whether you are submitting
the right file for the right problem and language and press `y' to
confirm. \cmd{submit} will report a successful submission or give
an error message otherwise.

The submit program uses a directory \cmd{\USERSUBMITDIR} in the
home directory of your account. In there, it stores temporary files for
submission and also a log file \cmd{submit.log}. Do not remove or
change this directory, or the \cmd{submit} program might fail to
function correctly. Furthermore a ``public ssh-key'' of the jury has
been added to the ssh configuration; this is also necessary for the
functioning of \cmd{submit}.

\subsection{Web-interface}

Solutions can be submitted from the web-interface at \url{\WEBBASEURI team}.
Navigate to \textbf{submit}, there you can select a file for
submission and the problem and language. These can also be left on the
default `automatic' setting; then the problem and language will be
determined from the base and extension of the filename respectively.

After you hit the submit button and confirm the submission, a page is
shown with a confirmation of successful upload. Next, you should be
able to see the submission in your submissions list.

\section{Viewing the results of submissions}

There is a overview of your submissions on your team web page.
It contains all relevant information: submission time, programming
language, problem and status. The address of your team page is
\url{\WEBBASEURI team}. From here you can also view the public
scoreboard page with the scores of all teams.

\subsection{Possible results}

A submission can have a few different results. All possible results
are listed here with a short description:

\begin{description}[\setleftmargin{4.5cm}]
\item[CORRECT]
The submission was correct and you solved this problem!

\item[COMPILER-ERROR]
There was an error when compiling your program. You can view the
submission details on your team page to see the exact error.
Notice the compiler\emph{warnings} do not count as error, but can also
be viewed in the submission details.

\item[TIMELIMIT]
Your program took longer than the maximum allowed time for this
problem. Therefore it has been aborted. This might indicate that your
program hangs in a loop, or that your solution is not efficient
enough.

\item[RUN-ERROR]
There was an error during the execution of your program. This can have
a lot of different causes like division by zero, incorrectly
addressing memory (e.g. by indexing arrays out of bounds), trying to
use more memory than the limit, etc\dots\ 
Also check that your program exits with exitcode 0!

\item[NO-OUTPUT]
Your program did not generate any output. Check that you write to
standard out.

\item[WRONG-ANSWER]
The output of your program was incorrect. This can happen, simply
because your solution is not correct, but also check that your output
complies exactly with the specifications of the jury.

\item[TOO-LATE]
Bummer, you submitted after the contest ended! Your submission is stored but
will not be processed anymore.
\end{description}

\section{Clarifications}

All communication with the jury is to be done with clarifications.
These can be found on your team page. Both clarification replies from
the jury and requests sent by you are displayed  there.

There is also a link to submit a new clarification request to the
jury. This request is only readable for the jury and they will
try to respond as soon as possible. Answers that are relevant for
everyone, will be sent to everyone.

When you receive a new clarification from the jury, you will
automatically be notified by a pop-up window. For this to work, you
have to have the team page opened in a web browser and the browser must
not block pop-up windows.

\section{How are submissions being judged?}

The \DOMjudge jury-system is fully automated. In principle no human
interaction is necessary. The judging is done in the following way:

\subsection{Submission}

With the \cmd{submit} (see \ref{submit}) program, you can submit a
solution to a problem. Note that you have to submit the source-code of
your program (and not a compiled program or the output of your
program). \cmd{submit} makes a connection to the submission server,
which copies your program to the jury-computers.

There your program enters a queue, awaiting being compiled, executed
and the output tested on one of the jury-computers.

\subsection{Compilation}

Your program will be compiled on a jury-computer running Linux.
Using a different compiler or operating system than the jury, should
not be a problem. Be careful however, not to use any special compiler
and/or system specific things (you can always check compiler errors on
the team page).

The jury-system compiles all programs statically. This should not
matter concerning generated errors.

\subsection{Testing}

After your program has compiled successfully, it will be executed and
it's output compared to the output of jury. Before comparing the
output, the exit status of your program is checked: if your programs
gives the correct answer, but exits with a nonzero exitcode, the
result will be a run-error! There are some restrictions during
execution. If your program violates these, it will also be aborted
with a run-error, see (\ref{runlimits}).

When comparing program output, it has to exactly match to output of
the jury. So take care, that you follow the output specifications. In
case of problem statements which do not have unique output (e.g. with
floating point answers), the jury can have made a modified comparison
function.

\subsection{Restrictions}\label{runlimits}

To prevent hacking, keep the jury-system stable and give everyone
clear and equal environments, there are some restrictions to which all
submissions are subjected:

\begin{description}[\setlabelphantom{number of processes}]
\item[compile time]
Compilation of your program may take no longer than \COMPILETIME\
seconds. After that, compilation will be aborted and results in a
compile error. In practice, this should never give rise to
problems. Should this happen to a normal program, please inform the
jury right away.

\item[source size]
The source-code of your program may not exceed \SOURCESIZE\ kilobytes.
When submitting your program, you will already be warned for this, but
during compilation this will result in a compile error.

\item[memory]
During execution of your program, there is \MEMLIMIT\ kilobytes memory
available. This is the total amount of memory (including program code,
statically and dynamically defined variables, stack, \dots)! If your
program tries to use more memory, it will (probably) abort, resulting
in a run error.

\item[filesize]
Although you are not allowed to write to files, standard (error)output
are written to a file. The maximal size of these files is \FILELIMIT\
kilobytes (per file).

\item[number of processes]
You are not supposed to create multiple processes (threads). This is
to no avail anyways, because your program has exactly 1 processor fully
at it's disposal. To increase stability of the jury-system, there is a
maximum of \PROCLIMIT\ processes that can be run simultaneously
(including processes that started your program).

People who have never programmed with multiple processes (or have
never heard of ``threads'') do not have to worry: a normal program
runs in one process.

\end{description}


\newpage
\appendix

\section{Code examples}\label{codeexamples}

Below are a few examples on how to read input and write output for a
problem. The way of doing reading and writing input and output in this problem
is like how it is done in a lot of programming contests (e.g. the ACM
regionals), but can be different at your contest.

The examples are solutions for the problem: given a string with a name
consisting of one word, output the string ``Hello $<$name$>$!''.

Sample input and output for this problem:

\begin{tabular}{|p{0.5\textwidth}|p{0.5\textwidth}|}
\hline
\textbf{Input} & \textbf{Output} \\
\hline
\verbatiminput{examples/example.in} &
\verbatiminput{examples/example.out} \\
\hline
\end{tabular}

Note that the number 3 on the first line indicates, that 3 testcases
follow.

A solution for this problem in C:
\listinginput{1}{examples/example.c}

Notice the last \cmd{return 0;} to prevent a run-error!

\newpage

A solution in C++:
\listinginput{1}{examples/example.cc}

A solution in Pascal:
\listinginput{1}{examples/example.pas}

\newpage

And finally a solution in Java:
\listinginput{1}{examples/example.java}

\end{document}
