\documentclass[11pt,a4paper,twoside]{article}

\usepackage{a4wide}
\usepackage{moreverb}
\usepackage{graphicx}
\usepackage{wrapfig}
\usepackage{expdlist}
\usepackage{svn}
\usepackage{fancyhdr}
\usepackage{extramarks}
\usepackage{color}
\usepackage[colorlinks=true,urlcolor=blue,linkcolor=blue]{hyperref}
\usepackage{ifthen}

\pagestyle{fancy}
\renewcommand{\sectionmark}[1]{\markboth{\thesection.\ #1}{}}

% Begin new paragraphs without indentation but vertical space.
\setlength{\parindent}{0pt}
\setlength{\parskip}{1.5ex plus 0.5ex minus 0.2ex}

\newcommand{\DOMjudge}{\textsc{DOM}judge }

% Display commands, arguments, etc. in texttt font and don't break those.
\newcommand{\cmd}[1]{\mbox{\texttt{#1}}}

% Only show text if the submitclient is configured
\newcommand{\ifcmdsubmit}[1]{\ifthenelse{\equal{\SUBMITCLIENTENABLED}{yes}}{#1}{}}

% Our titlepage, should be called at start of team manual document
% Argument list:
% #1 - DOMjudge version
% #2 - Document revision
% #3 - Last modified date
% #4 - Generated date
% Also define the following words for language overrides:
\newcommand{\versionrevison}{Version/revision}
\newcommand{\lastmodified}{Last modified}
\newcommand{\generated}{Generated}
\makeatletter
\newcommand{\titlestuff}[4]{%

  \thispagestyle{plain}
  \vspace*{-3cm}
  \parbox[t]{\linewidth}{%
    \begin{wrapfigure}[1]{r}{2cm}
      \vspace*{-1cm}\hfill
      \includegraphics[height=4cm]{../logos/DOMjudgelogo.pdf}
    \end{wrapfigure}
    {\fontfamily{phv}\fontseries{b}\fontsize{26pt}{28pt}\selectfont \@title \par}
  }
  \vskip 2cm

  % Setup fancy headers/footers (here because we need SVN stuff defined)
  \def\setupfancystuff{%
    \fancyhead{}
    \fancyfoot{}
    \fancyfoot[RO,LE]{\thepage}
    \fancyfoot[LO,RE]{%
      \color[gray]{0.5}\vspace{-0.3cm}
      \begin{tabular}{ll}
        \versionrevison: & #1 / #2 \\
        \lastmodified:   & #3 \\
        \generated:      & #4 \\
      \end{tabular}
    }
  }

  % First for fancy page style:
  \setupfancystuff
  \fancyhead[RO,LE]{\slshape \firstleftmark}

  % No headers for plain page style (titlepage):
  \fancypagestyle{plain}{%
    \setupfancystuff
    \renewcommand{\headrulewidth}{0pt}
  }
}
\makeatother


\usepackage[dutch]{babel}

% Redefine words for date/versioning:
\renewcommand{\versionrevison}{Versie/revisie}
\renewcommand{\lastmodified}{Laatst gewijzigd}
\renewcommand{\generated}{Gegenereerd}

% For inclusion of the correct lastmod date and revision:
\SVN $Date$
\SVN $Revision$

\title{\DOMjudge teamhandleiding}

\hypersetup{
	pdftitle={DOMjudge teamhandleiding},
	pdfauthor={DOMjudge Developers: domjudge-devel at lists.a-eskwadraat.nl},
	pdfsubject={Instructie voor teams die de interface van het DOMjudge jurysysteem gebruiken tijdens een programmeerwedstrijd},
	pdfkeywords={DOMjudge,manual,team,handleiding,judge,jury,programmeren,wedstrijd,programmeerwedstrijd,icpc,acm}
}

\begin{document}

\titlestuff{\DOMJUDGEVERSION}{\SVNRevision}{\SVNDate}{\today}

\section*{Samenvatting}

Hieronder staat de belangrijkste informatie kort samengevat. Dit is
bedoeld om snel aan de slag te kunnen. We raden echter ten zeerste
aan dat minstens \'e\'en iemand binnen je team de complete handleiding
doorneemt, omdat daarin specifieke details van het jurysysteem staan
die ook van belang kunnen zijn op het moment dat niet alles perfect
gaat. \textbf{WEES GEWAARSCHUWD!}

DOMjudge werkt via een web-interface die je kunt vinden op \\
\url{\WEBBASEURI team}.

\subsection*{Inlezen en wegschrijven}

Oplossingen moeten invoer en uitvoer lezen van `standard in'
(toetsenbord) en wegschrijven naar `standard out' (beeldscherm). Je
hoeft dus nooit een bestand te openen. Zie bijlage~\ref{codeexamples}
voor een aantal voorbeelden hiervan.

\subsection*{Insturen van oplossingen}

Insturen van oplossingen gaat via het command-line programma
\cmd{submit} dan wel de web-interface:
\begin{description}[\breaklabel\setlabelstyle{\bfseries}]
\item[Command-line] 
Gebruik \cmd{submit <probleem>.<extensie>}, met \cmd{<probleem>} het
label van het probleem en \cmd{<extensie>} een standaard-extensie van
de programmeertaal. Voor de complete documentatie en alle opties, zie \cmd{submit
--help}.
\item[Web-Interface]
Vanaf je teampagina op \url{\WEBBASEURI team}, selecteer je
\textbf{submit} en daar kun je een bestand selecteren en insturen.
Standaard wordt het probleem uit het deel van de bestandsnaam v\'o\'or de
punt gehaald en de programmeertaal uit de extensie.

\end{description}

\subsection*{Bekijken van scores, inzendingen, e.d.}

Het bekijken van inzendingen, scores en sturen en lezen van
``clarification requests'' gaat via de web-interface. De knoppen op
\url{\WEBBASEURI team} spreken voor zich.

\emph{Einde samenvatting.}

\newpage

\section{Oplossingen insturen}\label{submit}

Het insturen van oplossingen voor problemen kan op twee verschillende
manieren: via een command-line interface (het programma \cmd{submit}),
of via de web-interface. Het kan zijn dat een van beide niet beschikbaar
is, afhankelijk van de configuratie van het systeem door de jury.
Hieronder worden beide methodes beschreven.

\subsection{Command-line: \cmd{submit}}

\textbf{Syntax:} \cmd{submit [opties] bestandsnaam.ext}

Het submitprogramma haalt de naam (label) van het probleem uit
\cmd{bestandsnaam} en de de programmeertaal uit de extensie
\cmd{ext}. Dit kan handmatig aangepast worden met de opties
\cmd{-p probleemnaam} en \cmd{-l taalextensie}. Zie
\cmd{submit --help} voor een compleet overzicht van mogelijke
opties en extensies en een aantal voorbeelden. Als deze
helptekst niet op \'e\'en scherm past, gebruik dan
\cmd{submit --help | more} om alles te lezen.

\cmd{submit} zal het bestand controleren en eventueel
waarschuwingen geven, zoals wanneer het bestand al lange tijd niet
veranderd is of groter is dan de maximale source-code-grootte.

Hierna geeft \cmd{submit} een kort overzicht met de details van de
inzending en vraagt om bevestiging. Controleer vooral of je het goede
bestand, probleem en taal hebt en druk dan op `y' om de oplossing in
te sturen. Als alles goed gaat, zal \cmd{submit} een melding geven
dat de inzending succesvol is. Indien niet, zal er een foutmelding
verschijnen.

Het submitprogramma maakt gebruik van een directory \cmd{\USERSUBMITDIR}
in de homedirectory van het account. Hier slaat het tijdelijk
bestanden op voor inzending en staat ook een logfile \cmd{submit.log}.
Verwijder deze directory niet en pas hem niet aan, omdat anders het
submitprogramma niet meer correct functioneert. Verder kan er een
``public ssh-key'' van de jury in de \textsc{ssh}-configuratie
toegevoegd zijn. Ook deze is nodig voor het functioneren van \cmd{submit}.

\subsection{Web-interface}

Vanaf je teampagina \url{\WEBBASEURI team} kun je oplossingen insturen
door naar \textbf{submit} te gaan. Daar kan een bestand
geselecteerd worden om in te sturen. Verder kan het probleem en de
taal van de inzending ingesteld worden. Deze kunnen ook het
standaard geselecteerde `automatisch' blijven; dan wordt
geprobeerd het probleem en de taal van de inzending uit
respectievelijk de basis en de extensie van de bestandsnaam te halen.

Nadat je op de submitknop geklikt hebt en dit bevestigd hebt, wordt
aangegeven of je inzending goed aangekomen is. Daarna staat deze in je
lijst met inzendingen.

\section{De uitslag bekijken van inzendingen}

Op de team-webpagina staat een overzicht van je inzendingen.
Dit overzicht bevat alle relevante gegevens: de tijd van inzending, de
programmeertaal, het probleem en de status. Hier vind je ook het scorebord
met de resultaten van de andere teams.

\subsection{Mogelijke uitslagen}

Voor een ingestuurde oplossing zijn de volgende uitslagen mogelijk.

\begin{description}[\setleftmargin{4.5cm}]
\item[CORRECT]
Je oplossing heeft alle tests weerstaan: je hebt dit probleem opgelost!

\item[COMPILER-ERROR]
Het compileren van je programma gaf een fout. Bij de details
van deze inzending kun je de precieze foutmelding inzien
(deze optie kan uitgezet zijn).

\item[TIMELIMIT]
Je programma draaide langer dan de maximaal toegestane tijd en is
afgebroken. Dit kan betekenen dat je programma ergens in een loop
blijft hangen, of dat je oplossing niet effici\"ent genoeg is.

\item[RUN-ERROR]
Je programma gaf een fout tijdens het uitvoeren. Dit kan verschillende
oorzaken hebben, zoals deling door nul, incorrecte geheugen\-adressering
(segfault, bijvoorbeeld door arrays buiten bereik te indiceren), te
veel geheugengebruik, enzovoort.
Let ook op dat je programma met een exitcode 0 eindigt!

\item[NO-OUTPUT]
Je programma gaf geen uitvoer. Let op dat je uitvoer naar standard
output schrijft!

\item[WRONG-ANSWER]
De uitvoer van je programma was niet correct. Het kan zijn dat je
oplossing niet correct is, maar let ook goed op dat je de antwoorden
precies zoals beschreven uitvoert: de uitvoer moet exact kloppen met
de specificatie van de jury!

\item[PRESENTATION-ERROR]
De uitvoer van je programma verschilde slechts in presentatie
(bijvoorbeeld: de hoeveelheid witruimte) met de correcte uitvoer.
Dit wordt net als WRONG-ANSWER niet correct gerekend. Deze uitslag
is optioneel en mogelijk niet aangezet.

\item[TOO-LATE]
Helaas, je hebt ingestuurd nadat de wedstrijd al afgelopen was!
Je inzending is opgeslagen maar wordt niet verder behandeld.
\end{description}

\section{Clarifications}

Communicatie met de jury loopt door middel van \emph{clarifications}
(verhelderingen), deze komen op je teampagina te staan. Boven aan de
pagina de gegeven clarifications, daar onder je \emph{requests} (verzoeken).

Je kan vragen aan de jury stellen door middel van het doen van een
``Clarification Request'', de link hiervoor bevindt zich onderaan de
clarifications-pagina.  Je vraag zal alleen bij de jury aankomen; zij
zullen deze zo snel mogelijk en adequaat beantwoorden. Antwoorden die
voor iedereen relevant kunnen zijn zullen naar iedereen gestuurd worden.

Als je een clarification ontvangt van de jury, dan wordt dat
automatisch gemeld door toevoeging van ``(1 new)'' in de clarification
knop in het menu. Dit wordt automatisch bijgewerkt zonder dat het nodig
is de pagina te herladen.

\section{Hoe worden opgaven beoordeeld?}

Het \DOMjudge jurysysteem is volledig geautomatiseerd. Dit betekent
dat er (in principe) geen menselijke interactie is tijdens de
beoordeling. Het beoordelen gebeurt via de volgende stappen:

\subsection{Insturen}

Via het \cmd{submit} programma of de web-interface (zie
sectie~\ref{submit}) kun je een oplossing voor een opgave insturen,
zodat hij ge\"upload wordt naar de jury. Let op dat je de source-code
van je programma moet insturen (en dus niet een gecompileerd programma
of de uitvoer van je programma).

Dan komt je programma in de wachtrij te staan, om gecompileerd,
uitgevoerd en getest te worden op \'e\'en van de jury-computers.

\subsection{Compileren}

Je programma wordt op een jury-computer onder Linux gecompileerd.
Als je een andere compiler of besturingssysteem gebruikt dan de jury,
moet dat in principe geen probleem zijn, maar let wel op dat
je geen compiler/systeem-specifieke dingen gebruikt (afhankelijk van
de configuratie kun je bij een compileerfout de foutmelding bekijken).

\subsection{Testen}

Als je programma succesvol gecompileerd is, wordt het gedraaid en de
uitvoer vergeleken met de correcte uitvoer van de jury. Er wordt eerst
gecontroleerd of je programma correct ge\"eindigd is: als je programma
met een fout eindigt en het goede antwoord geeft, krijg je toch een
\textsc{run-error}! Er zijn een aantal beperkingen die aan je programma
opgelegd worden. Als je programma die overschrijdt, wordt het ook
afgebroken met een fout, zie sectie~\ref{runlimits}.

Verder moet de uitvoer van jouw programma exact overeenkomen met de
uitvoer van de jury. Let dus goed op, dat je de uitvoerspecificatie
volgt. In gevallen waarin er niet \'e\'en unieke uitvoer is (zoals bij
floating point-antwoorden) kan de jury een aangepaste beoordeling
hiervoor maken.

\subsection{Beperkingen}\label{runlimits}

Om misbruik tegen te gaan, het jurysysteem stabiel te houden en iedereen
duidelijke, gelijke omstandigheden te geven, zijn er een aantal
beperkingen die aan iedere ingestuurde oplossing opgelegd worden:

\begin{description}[\setlabelphantom{aantal processen}]
\item[compile-tijd]
Je programma mag er maximaal \COMPILETIME\ seconden over doen om te
compileren. Daarna wordt het compileren afgebroken en levert dit een
compileerfout op. Dit zou in de praktijk nooit een probleem mogen
opleveren. Mocht dit toch gebeuren bij een normaal programma, laat het
dan de jury weten.

\item[sourcegrootte]
De sourcecode van je programma mag maximaal \SOURCESIZE\ kilobytes
groot zijn, anders wordt je inzending geweigerd.

\item[geheugen]
Je programma heeft tijdens het draaien maximaal \MEMLIMIT\ kilobytes
geheugen ter beschikking. Let op dat dit totaal geheugen is (inclusief
programmacode, eventuele virtual machine (Java), statisch en dynamisch
gedefinieerde variabelen, stack, \dots)! Als je programma meer
probeert te gebruiken, zal het afgebroken worden, zodat dit een
``\textsc{run-error}'' geeft.

\item[uitvoergrootte]
Het is niet toegestaan meer dan \FILELIMIT\ kilobytes te schrijven naar
standard out of naar standard error. Als je deze limiet overschrijdt,
krijg je een \textsc{run-error}.

\item[aantal processen]
Het is niet de bedoeling dat je programma meerdere processen (threads)
start. Dit heeft ook geen zin, want je programma heeft precies \'e\'en
processor volledig tot zijn beschikking. Om de stabiliteit van het
jurysysteem te bevorderen, kun je maximaal \PROCLIMIT\ processen
tegelijk draaien (inclusief de processen waardoor je programma
gestart is).

Mensen die nooit met meerdere processen geprogrammeerd hebben (of
niet weten wat dat is), hoeven zich geen zorgen te maken: standaard
draait een gecompileerd programma in \'e\'en proces.

\end{description}

\subsection{Java klassenaamgeving}

Het compileren van Java broncode wordt gecompliceerd door de
klassenaamgeving van Java: er is geen vast startpunt van de code;
iedere klasse kan een methode \texttt{main} bevatten. Een klasse die
\texttt{public} gedeclareerd is, moet verder in een bestand met
dezelfde naam staan.

In de standaard configuratie detecteert DOMjudge automatisch de
hoofdklasse. Anders moet de hoofdklasse ``\verb!Main!'' heten en een
methode ``\verb!public static void main(String args[])!'' hebben. Zie
ook het Java codevoorbeeld in appendix~\ref{codeexamples}.

\newpage
\appendix

\section{Codevoorbeelden}\label{codeexamples}

Hieronder staan een aantal voorbeelden van code om de invoer van een
probleem in te lezen en de uitvoer weg te schrijven.

De code hoort bij de volgende probleembeschrijving: 
De invoer bestaat uit \'e\'en regel met daarop het aantal testgevallen.
Daarna volgt voor elk testgeval een regel met daarop een naam (\'e\'en
woord). Print voor elke naam de string ``Hello $<$naam$>$!''. Een naam
is maximaal 99 karakters lang.

Dit probleem zou de volgende in- en uitvoer kunnen hebben:

\begin{tabular}{|p{0.47\textwidth}|p{0.47\textwidth}|}
\hline
\textbf{Invoer} & \textbf{Uitvoer} \\
\hline
\verbatiminput{../examples/example.in} &
\verbatiminput{../examples/example.out} \\
\hline
\end{tabular}

Let op dat het getal 3 op de eerste regel aangeeft dat er 3
testgevallen volgen.

Een oplossing voor dit probleem in C:
\listinginput{1}{../examples/example.c}

Let op de \cmd{return 0;} aan het einde, zodat we geen
\textsc{run-error} krijgen!

\newpage

Een oplossing in C++ kan als volgt:
\listinginput{1}{../examples/example.cc}

Een oplossing in Java:
\listinginput{1}{../examples/example.java}

\newpage

Een oplossing in Pascal:
\listinginput{1}{../examples/example.pas}

En tenslotte een oplossing in Haskell:
\listinginput{1}{../examples/example.hs}

\end{document}
